\documentclass[uplatex, dvipdfmx, a4paper]{jsarticle}
\usepackage{reporttitle}
\usepackage{siunitx}
\usepackage{mhchem}
\usepackage{url}
\title{誘電性セラミックス}
\author{南川新明}
\date{\today}
\thema{令和5年度 機能材料実験 無機系}
\partner{秋村 寺谷 澤部}

\begin{document}
    \maketitle
    
    \section*{要旨}
    
    \section*{緒言}
        チタン酸バリウム(\ce{BaTiO3})は最初に発見された強誘電性セラミックスであり,
        コンデンサなどの材料として広く用いられている.チタン酸バリウムはペロブスカイト構造
        持つ.室温化では正方昌に属し,各イオンが理想的な位置から少し変移した対称中心のない構造
        をとる.このため正電荷と負電荷との電荷中心にずれが生じ,電気的に分極している.このような
        状態は自発分極と呼ばれる.この分極の方向は外部電場によって反転が可能であり,このような
        材料は強誘電体と呼ばれている.

        純粋なチタン酸バリウムは約\SI{120}{\celsius}で対称中心を持つ立方晶系に相転移し,
        それ以上の温度では自発分極のない常誘電体となる.相転移を起こす温度はCurie温度と
        呼ばれている.チタン酸バリウムの誘電率はCurie温度で極大を示し,それ以上の温度では
        Curie-Weiss の法則に従って低下する.また,チタン酸バリウムのCurie温度はバリウムあるいは
        チタンを種々の元素で置換することによりシフトさせることができる.本実験では
        バリウムの一部をストロンチウムで置換することで誘電特性の変化を検討した.


    \begin{thebibliography}{9}
        \bibitem{batio3-jstage}{強誘電体結晶(チタン酸バリウム同形体)}
            \\\url{https://www.jstage.jst.go.jp/article/nikkashi1898/59/11/59_11_1262/_pdf}\\
    \end{thebibliography}
\end{document}
